\chapter*{Sommario} %Se si cambia il Titolo cambiare anche la riga successiva così che appia corretto nell'indice

Questa tesi esplora l'interazione tra linguaggio naturale e giochi di carte collezionabili, focalizzandosi su \emph{Magic: The Gathering} come caso di studio. L'obiettivo è analizzare come in questi contesti diventi importante avere un linguaggio naturale ben strutturato e prevedibile. Il contributo principale riguarda l'applicazione di un Large Language Model per generare degli script per le carte. Attraverso questo approccio, si sono valutate le potenzialità e le limitazioni dei modelli di linguaggio avanzati nell'ambito de generazione script utilizzando un linguaggio a dominio specifico. Sono stati eseguiti diversi esperimenti ed i migliori risultati quantitativi e qualitativi si sono ottenuti con uno Small Language Model.
Grazie a questi esperimenti, sarà possibile formalizzare il linguaggio nella sua interezza e verrà reso disponibile lo Small Language Model per la generazione di script delle carte.